\documentclass[journal=jacsat,manuscript=communication]{achemso}
\mciteErrorOnUnknownfalse
% \documentclass{achemso}
\usepackage{mhchem}
\usepackage{siunitx}
\usepackage{graphicx}
\usepackage{xcolor}
\usepackage{float}

\title{Coding a Variational Method \& Hartree-Fock solver using Open-Source tools.}
\author{Adam Keim}
\email{a_keim@coloradocollege.edu}

\newcommand*\dif{\mathop{}\!\mathrm{d}}
\newcommand{\matr}[1]{\mathbf{#1}}

\begin{document}
\begin{abstract}
Understanding quantum mechanics principles is crucial to chemistry education, yet often overlooked due to its complexity.  This work aims to bridge this gap by experimenting with open-source computational methods to elucidate quantum mechanics concepts in the classroom.  Leveraging elementary variational methods, as well as more advanced Hartree-Fock methods, implemented in Python, students are guided through solving quantum mechanical systems.  Results are compared to exact values, showing both the efficacy and limitations of these approaches.  Further directions include generalizing these methods to larger systems, as well as exploring modern techniques such as density functional theory.  \end{abstract}
\section{Introduction}
Most undergraduate chemistry curriculums gloss over underlying quantum mechanical principles, in large part due to the amount of mathematics required, as well as how abstract many of the ideas involved are. Despite not receiving instruction in quantum mechanics (QM) past toy systems, such as the classic (not classical!) particle in a box, many of these same students are exposed to software packages like Gaussian, which harness the aforementioned QM principles to perform physical chemistry calculations.  By leveraging a number of computational approaches, this project aims to show the value of computational quantum methods in the classroom as a way to make computational solver packages less of a "black box."

The project presented in this work uses open-source computational chemistry methods to guide students through solving a number of quantum-mechanical systems, using both variational and Hartree-Fock methods.  All of the code is implemented in Python, and is based on the work of \citet{morgensternUnderstandingQuantumMechanical2023}, \citet{schrierIntroductionComputationalPhysical2017}.  In each case, the calculated values are compared to exact values from the literature, and in the case of the more advanced methods, further directions that students may want to take these projects are emphasized.


\section{Background}
There are a number of assumptions that we will make in all of the following cases to drastically simplify solving our system.  The first, and most important of these is that we will neglect relativistic effects, treating our momentum operator as non-relativistic.  We also employ the Born-Oppenheimer approximation, a standard assumption employed in most molecular models, which treats the positions of nuclei as (fixed) parameters for the system rather than variables.  There are a number of other assumptions which are made, which relate to how we describe wavefunctions as linear combinations of orthogonal basis functions, and that each stationary state (energy eigenfunction) of the system corresponds to one orbital.  We also note that when we are discussing the Hartree-Fock method, our discussion only applies to closed-shell systems (restricted Hartree-Fock), where all orbitals are doubly occupied. One can extend Hartree-Fock methods to consider open-shell, unrestricted systems, but that is beyond the scope of this work. 

When we employ these assumptions, the limits on the simple variational method are dictated by our choice of trial wavefunction, and the limit of accuracy on the Hartree-Fock method is known as the Hartree-Fock limit, and is based on limitations inherent to the method which we will discuss. 

There are two methods that we will use to estimate energy levels for the systems we are considering.  The first of these is the variational method, a foundational tool in quantum mechanics (a detailed explanation, and justification for, this method can be found in any quantum mechanics textbook, though I am quite fond of Griffiths\cite{griffithsIntroductionQuantumMechanics2018}).  We will apply this method to a 2-nucleus \ce{H2+} molecular ion, and calculate energy levels and a bond dissociation curve.  The goals of this experiment are to lay out some of the foundational software tools that we will be using for some of the other systems, as well as showing how close of an estimate one can arrive at using quite simple methods.

The second method is the Hartree-Fock method, which allows us to consider multiple electrons (as long as orbitals are filled, remember this is \textbf{Restricted} Hartree-Fock).  This method is based on the variational method but differs both in how it describes orbitals (basis sets instead of trial functions), and in its use of iterative calculations (hence why these methods are sometimes referred to as self-consistent field (SCF) solvers).  We will first apply this method to a simple He atom, and then extend it to an \ce{H2} molecule.  

\section{Experimental}
\subsection{Variational solution of a 2 Nucleus system}
The first choice that needs to be made to analyze the two nucleus \ce{H2+} molecular ion is that of a suitable coordinate system.  In this work, confocal ellipsoidal coordinates are used, where the two foci of the ellipse are given by the two nuclei.  There are many choices of variable symbols for this coordinate system, but in this work we follow the convention of \citet{morgensternUnderstandingQuantumMechanical2023}.  Surfaces of constant $R$, $\xi$, and $\eta$ are shown in Figure \ref{fig:confocal}.

\begin{figure}[H]
  \includegraphics{figures/confocal}
  \caption{Surfaces of constant $R$, $\eta$ and $\xi$.  Figure adapted from \citet{weissteinConfocalEllipsoidalCoordinates}.}
  \label{fig:confocal}
\end{figure}
Now we set up our system and describe it using tools from quantum mechanics.
\begin{figure}[H]
	\includegraphics[width=.3\textwidth]{figures/hydrogenmolecule.png}
	\caption{A schematic of our hydrogen molecular ion system.  The $z$-axis is not relevant here, since we are working in confocal coordinates. Figure adapted from \citet{fitzpatrickHydrogenMoleculeIon}}
\end{figure}
The two protons act as the two foci of our ellipsoid, and we can describe the position of the electron as a single point in confocal ellipsoidal space.  Working out the math for this gives a Hamiltonian (an operator from quantum mechanics that describes the total energy of a system) of
\begin{equation}
	\hat{H} = \frac{-2}{R^2\left(\xi^2-\eta^2\right)}\left[\left(\xi^2-1\right)\frac{\partial^2 }{\partial\xi^2}\right]-\frac{2}{R(\xi+\eta)}-\frac{2}{R(\xi-\eta)}+\frac{1}{R}.
\end{equation}
in atomic units.  Atomic units are quite covenient for a problem like this, since they make many of the coeffiicients one often has to consider in these circumstances become simple integers.  We note that the energy of the system depends only on its position (defined in $(R,\eta,\xi)$ coordinates).

To apply the variational method, a trial wavefunction needs to be selected, and in this case a reasonable choice is
\begin{equation}
	\phi_t = c\left(\frac{Z^{3/2}}{\sqrt{\pi}}e^{-Zr_a} + \frac{Z^{3/2}}{\sqrt{\pi}}e^{-Zr_b}\right),
\end{equation}
where c is a constant.  This is the sum of trial wavefunctions often used for hydrogenic atoms, where $Z$ represents the effective charge of the nucleus, $r_a,r_b$ describe the distance from each nucleus.
  
With the Hamiltonian and a trial wavefunction, we can apply the variational method.  Since our trial wavefunction is not necessarily normalized, we claim that the ground state energy $E_\textrm{ground}$ follows
\begin{equation}
E_\textrm{ground} \leq \frac{\left<\phi | \hat{H} | \phi \right>}{\left<\phi|\phi\right>}.
\end{equation} 
The angular bracket notation is known as bra-ket notation and is a standard notation commonly used in quantum mechanics, where
\begin{align}
	\left<x|y\right> &= \int x^* y d\tau \\
	\left<x|\hat{O}|y\right> &= \int x^* \hat{O}yd\tau.
\end{align}
We end up with a number of relatively simple equations which we can plot to arrive at the graphs shown in Figure \ref{fig:contour} and Figure \ref{fig:var_bd}
\subsection{Self-consistent calculations for \ce{He} atom}
For an \ce{He} atom, things get a little more complicated.  There are no known analytic solutions for systems of more than one electron, so even the best solutions still have errors in their approximations.  Where the variational method uses the Hamiltonian of the system and a trial wavefunction to describe the quantum system, Hartree-Fock uses  and we can use Slater-type \textit{1s} functions with the form 
\begin{equation}
	\phi_n(r,\theta,\phi) = 2\zeta_n^{3/2}e^{-\zeta_nr/a_0}Y_0^0(\theta,\phi)
\end{equation}
where $\zeta$ is a parameter that describes the spatial extent of the function, $a_0$ is the Bohr Radius, and $Y_0^0$ is the first spherical harmonic.  Optimal values for $\zeta$ are selected according to \citet{roettiSimpleBasisSets1974}.  The method used to evaluate the wavefunction is known as the Hartree-Fock method, and consists of the following major steps:
\begin{enumerate}
	\item Choose the number of occupied orbitals $N_\mathrm{occ}$, number ($N_\mathrm{basis}$), and type of bases.
	\item Calculate overlap, orthogonality transform, one-electron, and two-electron matrices.  For both of our choices of basis set, we can calculate these analytically ahead of time rather than numerically.
	\item Make initial guesses for eigenvector coefficients and use these to calculate a charge density matrix, given by
\begin{equation}
	P_{tu} = 2 \sum_{i=1}^{N_{\mathrm{occ}}}c^*_{it}c_{iu}.
\end{equation}
	This matrix describes where the electrons are located in terms of our bases.  Before we've gotten good values for $c_{ij}$ from Step 6, a reasonable guess for $c$, and thus $P$, is 0.  
	\item Build our Fock matrix $F_{rs}$ following

\begin{equation}
	F_{rs} = h_{rs} + \sum_{t=1}^{N_\mathrm{basis}}\sum_{u=1}^{N_\mathrm{basis}}P_{tu}\left\{(rs|tu) -\frac{1}{2}(ru|ts)\right\}.
\end{equation}
	\item Calculate $\matr{F}' = \matr{X}^\dag\matr{F}\matr{X}$.
	\item Solve the eigensystem $\matr{F}'\vec{c'}_i=\epsilon_i\vec{c'}_i$.
	\item Transform back to our original basis $\vec{c}_i$
	\item Build a new charge density matrix, $\matr{P}$ from eigenvectors obtained in previous step.
	\item If the difference in energy 
\begin{equation}
	E = \sum_{i=1}^{N_\mathrm{occ}}\epsilon_i + \frac{1}{2} \sum_{r=1}^{N_\mathrm{basis}} \sum_{s=1}^{N_\mathrm{basis}}P_{rs}h_{rs}+V_{nn}
	\label{eq:hf_energy}
\end{equation}
is smaller than some threshold, terminate.  Otherwise, go back to step 4.
\end{enumerate}

Practically, we don't define a threshold for Step 9, since the systems we are interested are small enough that they converge after $<10$ iterations.  If our system does not converge, we will see "spikes" in the graph in Figure \ref{fig:hf_bd}\cite{schrierIntroductionComputationalPhysical2017}.


\subsection{Self-consistent calculations for \ce{H2}}
Once there are two nuclei in the system, Slater-type orbital functions get messy, so we approximate the orbital wavefunctions as a linear combination of Gaussian-type orbitals, to both speed up and simplify the following calculations.  A $1s$ Gaussian basis function is given by
\begin{equation}
	g_{1s}(\vec{r}_\alpha)=\left(\frac{2\alpha}{\pi}\right)^{3/4}e^{-\alpha|\vec{r}_\alpha|^2},
\end{equation}
which has the familiar form of a Gaussian distribution, and is quite simple to integrate analytically.  The only other thing that changes relative to the previous experiment, other than basis sets, is that there is now a contribution to the energy from the Coulombic repulsion between the two nuclei, which must be taken into account. (The $V_{nn}$ term in Equation \ref{eq:hf_energy})  Otherwise, the self-consistent field calculations remain similar, and we are able to calculate a ground state energy and a bond dissociation curve.
\section{Results and Discussion}

{\setlength{\extrarowheight}{2pt}
\begin{table}[H]
  \begin{tabular}{l|ccc}
    Property & Variational Method & Exact & \% Error \\
    \hline
    $R_e$ (\si{\angstrom}) & 1.060 & 1.057 & .28\%\\
    $E$ (\si{\hartree}) & -.5865 & -.6026 & 2.67\%
  \end{tabular}
  \caption{Results for the variational method applied to \ce{H2+}.  Exact values from \citet{morgensternUnderstandingQuantumMechanical2023}.  $R_e$ is the equilibrium bond length, and $E$ is the ground state energy. }
  \label{tab:var}
\end{table}

Table \ref{tab:var} compares the values obtained from the variational method to exact values from the literature, and we see that the values are in quite good alignment, to within $<3\%$. 
%TODO: FIX THIS
 Unfortunately, the variational method is limited to systems of one electron, though we can still obtain an energy contour plot showing the variational energy across $R,Z$ space.  By taking the minimum energy for each $R$, we obtain a bond dissociation curve (Figure \ref{fig:var_bd}), which has the characteristic shape that we expect to see for a curve of this type\cite{atkinsAtkinsPhysicalChemistry2018}.

\begin{figure}[H]
  \includegraphics[width=.5\textwidth]{figures/variational_energy_contour.png}
  \caption{Energy contour plot for the \ce{H2+} molecular ion, calculated with the variational method.}
  \label{fig:contour}
\end{figure}



\begin{figure}[H]
  \includegraphics[width=0.5\textwidth]{figures/variational_dissociation.png}
  \caption{The bond dissociation curve for the \ce{H2+} molecular ion, calculated with the variational method.}
  \label{fig:var_bd}
\end{figure}

{\setlength{\extrarowheight}{2pt}
\begin{table}[H]
  \begin{tabular}{l|ccc}
    Property & Hartree-Fock & Exact & \% Error \\
    \hline
    $R_e$ (\si{\angstrom}) & .78 & .74 & 5.4\%\\
    $E$ (\si{\hartree}) & -.9738 & -1.1744 & 17\%
  \end{tabular}
  \caption{Results for the Hartree-Fock method applied to \ce{H2}.  Exact values from \citet{schrierIntroductionComputationalPhysical2017}.  $R_e$ is the equilibrium bond length, and $E$ is the ground state energy. }
  \label{tab:hf}
\end{table}

\begin{figure}[H]
  \includegraphics[width=0.5\textwidth]{figures/Bond_Dissociation_Hydrogen.png}
  \caption{The bond dissociation curve for the \ce{H2} molecule, calculated with Hartree-Fock methods.}
  \label{fig:hf_bd}
\end{figure}

\section{Conclusion}
Molecular modeling has been shown to increase understanding of physical chemistry concepts\cite{ealyStudentsUnderstandingEnhanced2004}, though the underlying mechanics are rarely covered in chemistry curriculums.  Abstract concepts such as quantum mechanics lend themselves particularly well to computational methods, given the heavy mathematics and non-intuitive concepts involved.  By providing students with a sequence of experiments which they can carry out in the classroom, this work aims to "de-mystify" some of the quantum-motivated computations happening inside of computational chemistry packages such as Gaussian. The tools used for the computations are exclusively open-source, with hopes that students who do not have access to, or do not wish to use, proprietary software packages like Mathematica can study and replicate these experiments as an method to further their chemistry education. Overall, this project aims to point students at a number of projects they can carry out, as well as providing Python code to assist in debugging and verifying results. The hope is that students interested in understanding the functionality of these advanced algorithms will be able to refer to \citet{schrierIntroductionComputationalPhysical2017}, \citet{morgensternUnderstandingQuantumMechanical2023} as well as my code, to further their understanding, but the best way to understand a concept like this will always be to code it yourself, looking at my code as minimally as possible. 

\section{Further Directions}
There are a number of quite interested further directions for this research.  Interested students could generalize the Hartree-Fock code to either treat systems with more numerous nuclei or electrons, or remove the restriction on our Hartree-Fock implementation that forces completely full orbitals.  It would also be interesting to add more complex gaussian basis sets to better describe the spatial extent of the wavefunction of electrons, recalling that this will approach an asymptotic Hartree-Fock limit.  Modern approaches to similar problems often either exclusively use, or combine with Hartree-Fock, methods known as density functional theory (DFT), which is an exact method that does not have an intrinsic limit, and is limited by our approximations about physical systems.  
\nocite{*}
\bibliography{bib.bib}
\end{document}